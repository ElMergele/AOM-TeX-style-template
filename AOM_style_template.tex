% This is a very simple template that roughly follows the Academy of Management style guide.
% There remain a number of areas for improvement, so please submit updates and fixes to the GitHub repository.
% Note: you must use XeLaTeX or similar to typeset this document.

\documentclass[12pt,letterpaper]{article}

  \usepackage{fontspec}
  \setmainfont{Times New Roman}
  
  % Define 1-inch page margins
  \usepackage[margin=1in]{geometry}
  
  % Double spacing for main text, but leaves footnotes as single spacing:
  \usepackage{setspace}
  \doublespacing
  
  % Places the submission number in upper right header
  \usepackage{fancyhdr}
  \pagestyle{fancy}
  \fancyhf{} % clears header and footer
  \renewcommand{\headrulewidth}{0pt}
  \rhead{Submission ID: 00000}
  
  % Places all tables at end of document and creates AOM-style table-here placeholders
  \usepackage[nolists]{endfloat} % Places all figures and charts at end of manuscript and adds 'insert table x about here' lines.
  \renewcommand{\figureplace}{
    \begin{center}
    \begin{singlespace}
    ------------------------------------\\
    Insert \figurename \ \thepostfig\ about here.\\
    ------------------------------------
    \end{singlespace}
    \end{center}}
  \renewcommand{\tableplace}{%
    \begin{center}
    \begin{singlespace}
    ------------------------------------\\
    Insert \tablename \ \theposttbl\ about here.\\
    ------------------------------------
    \end{singlespace}
    \end{center}}

  % Bold Table and Figure captions
  \usepackage{caption}
  \captionsetup{figurename=FIGURE}
  \captionsetup{tablename=TABLE}
  \captionsetup[figure]{labelfont=bf}
  \captionsetup[table]{labelfont=bf}
  
  % Turns off all section numbering
  \setcounter{secnumdepth}{0} 
  
  % Makes the sections and subsections the same font size as normal document text; section titles are bold, uppercase, and centered; subsections are just bold
  \usepackage{titlesec}
  \titleformat{\section}
    {\filcenter\normalfont\bfseries\uppercase}{\thesection}{1em}{}
  \titleformat{\subsection}
    {\normalfont\bfseries}{\thesubsection}{1em}{}
  \titleformat{\subsubsection}
    {\normalfont\bfseries}{\thesubsubsection}{1em}{}

  \usepackage[hidelinks]{hyperref}
  \usepackage[round]{natbib}
  
  % Justifies paragraphs in standard flush left format
  \raggedright

% Document content begins here:
    
\begin{document}

\centerline{\textbf{The Title of the Paper}}

\section{Abstract}
Lorem ipsum dolor sit amet, consectetur adipiscing elit, sed do eiusmod tempor incididunt ut labore et dolore magna aliqua. Ut enim ad minim veniam, quis nostrud exercitation ullamco laboris nisi ut aliquip ex ea commodo consequat. Duis aute irure dolor in reprehenderit in voluptate velit esse cillum dolore eu fugiat nulla pariatur. Excepteur sint occaecat cupidatat non proident, sunt in culpa qui officia deserunt mollit anim id est laborum.

\textbf{Keywords:} keyword1, keyword2, keyword3, keyword4, keyword5

\newpage
	
\section{Introduction}

Lorem ipsum dolor sit amet, consectetur adipiscing elit, sed do eiusmod tempor incididunt ut labore et dolore magna aliqua. Ut enim ad minim veniam, quis nostrud exercitation ullamco laboris nisi ut aliquip ex ea commodo consequat. Duis aute irure dolor in reprehenderit in voluptate velit esse cillum dolore eu fugiat nulla pariatur. Excepteur sint occaecat cupidatat non proident, sunt in culpa qui officia deserunt mollit anim id est laborum.

\subsection{This is a subsection}

Lorem ipsum dolor sit amet, consectetur adipiscing elit, sed do eiusmod tempor incididunt ut labore et dolore magna aliqua. Ut enim ad minim veniam, quis nostrud exercitation ullamco laboris nisi ut aliquip ex ea commodo consequat. Duis aute irure dolor in reprehenderit in voluptate velit esse cillum dolore eu fugiat nulla pariatur. Excepteur sint occaecat cupidatat non proident, sunt in culpa qui officia deserunt mollit anim id est laborum. See Table \ref{tab:data_table} for a summary.

\begin{table}[h]
\caption{This is a description of some data.}
\label{tab:data_table}
\centering
\begin{tabular}{|l|l|l|}
\hline
A & B & C \\ \hline \hline
1 & 2 & 3 \\ \hline
x & y & z \\ \hline
\end{tabular}
\end{table}

\section{The second section}

Lorem ipsum dolor sit amet, consectetur adipiscing elit, sed do eiusmod tempor incididunt ut labore et dolore magna aliqua. Ut enim ad minim veniam, quis nostrud exercitation ullamco laboris nisi ut aliquip ex ea commodo consequat. Duis aute irure dolor in reprehenderit in voluptate velit esse cillum dolore eu fugiat nulla pariatur. Excepteur sint occaecat cupidatat non proident, sunt in culpa qui officia deserunt mollit anim id est laborum.

\end{document}
